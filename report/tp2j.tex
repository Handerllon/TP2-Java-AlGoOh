\input{config.tex}
%-----------------------------------%
%									%
%		Comienzo del documento		%
%									%
%-----------------------------------%
\begin{document}
%-----------------------------------%
%									%
%			Caratula				%
%									%
%-----------------------------------%
\pagestyle{fancy}
\begin{titlepage}
	\newcommand{\HRule}{\rule{\linewidth}{0.5mm}} % Defines a new command for horizontal lines, change thickness here
	\center % Centre everything on the page
	
	\thispagestyle{empty}
	\begin{center}
		\includegraphics[scale=1]{includes/banner_fiuba.pdf}\\
	\end{center}

	\vspace*{\stretch{1}}
	
	\textsc{\LARGE \textsc{[75.07] Algoritmos y Programación III}}
	\\[0.5cm]
	\textsc{\large 1\textsuperscript{o} Cuatrimestre 2018}
	\\[0.5cm]
	\textsc{\large Turno noche}
	\\[0.5cm]
	
	\HRule
	\\[0.5cm]
	{\huge\bfseries TP2: Al-Go-Oh}
	\\[0.2cm]
	\HRule
	\\[0.5cm]
	
	\begin{tabbing}
		\hspace{2cm}\=\+
		\underline{AUTORES}\hspace{-1cm}\=\+\hspace{1cm}\=\hspace{6cm}\=\\
		\\
		Anderson, Manuel			\>\>- \#101.230\\
		\>\footnotesize{$<$manuel121097@gmail.com$>$}\\
		\\
		Arredondo, Nicolás			\>\>- \#95.618\\
		\>\footnotesize{$<$nicolas\_arredondo@hotmail.com$>$}\\
		\\
		Husain, Ignacio Santiago	\>\>- \#90.117\\
		\>\footnotesize{$<$santiago.husain@gmail.com$>$}\\
		\\
		Parente, Gastón			 	\>\>- \#101.516 \\
		\>\footnotesize{$<$ggparente95@gmail.com$>$}\\
		\\
		\<\underline{DOCENTES}\\
		\\
		Lic. Suarez, Pablo \\
		\\
		Ing. Diego, Sánchez \\
		\\
		Srta. Marijuán, Magalí\\
		\\
		Sr. Leal Bazterrica, Matías
	\end{tabbing}

	\vspace*{\stretch{1}}

	\today

\end{titlepage}

\clearpage
\tableofcontents
%-------------------------------%
%								%
%			Seccion				%
%								%
%-------------------------------%
\clearpage
\section{Objetivo del trabajo}

\section{Supuestos}

[Documentar todos los supuestos hechos sobre el enunciado. Asegurarse
de validar con los docentes]

\section{Modelo de dominio}

[Explicar los elementos más relevantes del diseño. Es decir: qué
entidades se han creado, qué responsabilidades tienen asignadas, cómo
se relacionan, etc]

\clearpage
\section{Diagramas de clases}

[ Varios diagramas de clases, mostrando la relación estática entre las
clases, pueden agregar todo el texto necesario para aclarar y explicar su
diseño, recuerden que la idea de todo el documento es que quede
documentado y entendible como está hecho el TP]

\begin{figure}[H]
	\centering
	\includegraphics[scale=0.9]{includes/AlGoOh}
	\caption{titulo.}
	\label{AlGoOh}
\end{figure}

\begin{figure}[H]
	\centering
	\includegraphics[scale=0.9]{includes/Tablero}
	\caption{titulo.}
	\label{Tablero}
\end{figure}

\begin{figure}[H]
	\centering
	\includegraphics[scale=0.9]{includes/Cartas}
	\caption{titulo.}
	\label{Cartas}
\end{figure}

\begin{figure}[H]
	\centering
	\includegraphics[scale=0.9]{includes/Implementacion_de_efectos}
	\caption{titulo.}
	\label{Implementacion_de_efectos}
\end{figure}

\begin{figure}[H]
	\centering
	\includegraphics[scale=0.9]{includes/Implementacion_cartas_magicas}
	\caption{titulo.}
	\label{Implementacion_cartas_magicas}
\end{figure}

\begin{figure}[H]
	\centering
	\includegraphics[scale=0.9]{includes/Implementacion_carta_monstruo}
	\caption{titulo.}
	\label{Implementacion_carta_monstruo}
\end{figure}

\clearpage
\section{Diagramas de secuencia}

[ Varios diagramas de secuencia, mostrando la relación dinámica entre las
clases planteando una gran cantidad de escenarios que contemplen las
situaciones del trabajo práctico]

\begin{figure}[H]
	\centering
	\includegraphics[scale=0.9]{includes/AtacarCasoMonstruoConMenosAtaqueAMonstruoConMasAtaque}
	\caption{titulo.}
	\label{AtacarCasoMonstruoConMenosAtaqueAMonstruoConMasAtaque}
\end{figure}
\begin{figure}[H]
	\centering
	\includegraphics[scale=0.9]{includes/JugadorJuegaMonstruoQueRequiereSacrificio}
	\caption{titulo.}
	\label{JugadorJuegaMonstruoQueRequiereSacrificio}
\end{figure}
\begin{figure}[H]
	\centering
	\includegraphics[scale=0.9]{includes/JugadorUsaCartaAgujeroNegro}
	\caption{titulo.}
	\label{JugadorUsaCartaAgujeroNegro}
\end{figure}

\section{Diagramas de paquetes}

[incluir un diagrama de paquetes para mostrar el acoplamiento de su
trabajo ]

\section{Diagramas de estado}

[Incluir diagramas de estados, mostrando tanto los estados como las
distintas transiciones de los mismos para varias entidades del trabajo
práctico ]

\section{Detalles de implementación}

[Deben detallar/explicar qué estrategias utilizaron para resolver todos
los puntos más conflictivos del trabajo práctico. ]

\section{Excepciones}

[Explicar las excepciones creadas, con qué fin fueron creadas y cómo y
dónde se las atrapa explicando qué acciones se toman al respecto una vez
capturadas.]

%-------------------------------%
%								%
%			Seccion				%
%								%
%-------------------------------%
\appendix
\section{Referencias}
% Removes 'Referencias' title from 'thebibliography'.
\begingroup
\renewcommand{\section}[2]{}
\begin{thebibliography}{10}
	\bibitem{libro_fontela1} Fontela, Carlos - \emph{Programación Orientada a Objetos con Smalltalk, Java y UML.} - 3\textsuperscript{ra} edición - Versión Beta 0.6.
	
	\bibitem{fowler_model} Fowler, M. - \hyperref{https://martinfowler.com/distributedComputing/purpose.pdf}{}{}{What's a model for?}
\end{thebibliography}
\endgroup

\end{document}
